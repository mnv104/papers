% -*- TeX-master: "paper.tex"; TeX-PDF-mode: t; ispell-local-pdict: "words" -*-

File systems hosting virtual machines typically contain many
duplicated blocks of data resulting in wasted storage space and
increased storage array cache footprint. \emph{Deduplication}
addresses these problems by
storing a single instance of each unique data block and sharing it
between all original sources of that data.  While deduplication is
well understood for file systems with a centralized component, we
investigate it in a
%decentralized storage area network
decentralized cluster file system, specifically in the context of
VM storage.

% where multiple hosts symmetrically share access to a common disk
% array.

We propose \DeDe, a block-level deduplication
system for live cluster file systems that does not require any central
coordination, tolerates host
failures, and takes advantage of the block layout policies of an
existing cluster file system.
%
In \DeDe, hosts keep summaries of their own writes to the cluster file
system in shared on-disk logs. Each host periodically and
independently processes the summaries of its locked files, merges them
with a shared index of blocks, and reclaims any duplicate blocks.
\DeDe manipulates metadata using general file system
interfaces without knowledge of the file system implementation.
%
We present the design, implementation, and evaluation of our techniques
in the context of VMware ESX Server.  Our results show an 80\%
reduction in space with minor performance overhead for realistic
workloads.
